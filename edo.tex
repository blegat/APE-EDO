\documentclass[fontsize=12pt]{article}

\usepackage[utf8x]{inputenc}
\usepackage[frenchb]{babel}
\usepackage[T1]{fontenc}
\usepackage{lmodern}
\usepackage[top=2cm, bottom=2cm, left=2.5cm, right=2.5cm]{geometry}
\usepackage{graphicx}
\usepackage{epstopdf}
\usepackage{caption}
\usepackage{subcaption}
\usepackage{multirow}

% Math symbols
\usepackage{amsmath}
\usepackage{amssymb}
\usepackage{amsthm}

% Numbers and units
\usepackage[squaren, Gray]{SIunits}
\usepackage{sistyle}
\usepackage[autolanguage]{numprint}
%\usepackage{numprint}
\newcommand\si[2]{\numprint[#2]{#1}}
\newcommand\np[1]{\numprint{#1}}

\DeclareMathOperator{\newdiff}{d} % use \dif instead
\newcommand{\dif}{\newdiff\!}
\newcommand{\fpart}[2]{\frac{\partial #1}{\partial #2}}
\newcommand{\ffpart}[2]{\frac{\partial^2 #1}{\partial #2^2}}
\newcommand{\fdpart}[3]{\frac{\partial^2 #1}{\partial #2\partial #3}}
\newcommand{\fdif}[2]{\frac{\dif #1}{\dif #2}}
\newcommand{\ffdif}[2]{\frac{\dif^2 #1}{\dif #2^2}}
\newcommand{\constant}{\ensuremath{\mathrm{cst}}}

\newcommand{\vmin}{V_\mathrm{min}}
\newcommand{\vmax}{V_\mathrm{max}}
\newcommand{\tmax}{T_\mathrm{max}}
\newcommand{\dmax}{D_\mathrm{max}}
\newcommand{\tdmin}{TD_\mathrm{min}}
\newcommand{\vtmax}{VT_\mathrm{max}}
\newcommand{\vdmax}{VD_\mathrm{max}}
\newcommand{\mmax}{M_\mathrm{max}}
\newcommand{\cjour}{C_\mathrm{jour}}

\usepackage{xparse}% for using parameters at the end block

% Inclure une image
\NewDocumentEnvironment{image}{mm}
{\begin{figure}[!ht]\begin{center}}
{\caption{#2}\label{#1}\end{center}\end{figure}}

\newcommand{\showimage}[3]
{\begin{image}{fig:#2}{#3}\centering
\includegraphics[width=\textwidth]{#1}\end{image}}

% Inclure in plot matlab
\usepackage{xargs}
\newcommand{\matlabplot}[2]
{\showimage{Images/#1.eps}{#1}{#2 Généré par le listing~\ref{lst:#1}}.}
\newcommand{\matlabcode}[2]
{\lstinputlisting[caption={Contenu du fichier \lstinline{#1.m}
  contenant l'implementation de la fonction \lstinline{#1}.
#2},label={lst:#1}]
{MATLAB/#1.m}}

\usepackage[usenames,dvipsnames,svgnames,table]{xcolor}
\usepackage{listings}
\definecolor{dkgreen}{rgb}{0.25,0.7,0.35}
\definecolor{dkred}{rgb}{0.7,0,0}
\lstset{language={Matlab},numbers=left,numberstyle=\tiny\color{gray},
keywordstyle=\bfseries\color{dkred},frame=single,
commentstyle=\color{gray}=small,stringstyle=\color{dkgreen},
basicstyle=\rm\footnotesize\ttfamily} % basicstyle=\rm\footnotesize

\newcommand{\matlab}{\textsc{Matlab}}

\title{Correctif EDO}
\author{Benoît Legat}

\begin{document}

\maketitle

Les corrections de ce correctif seront soumises au prof d'EDO pour
être rajoutée dans le syllabus de l'an prochain.
En attendant, elles sont publiées par ce biais ci.
Vous êtes bien entendu invités à en ajouter.

\section{Systèmes linéaires à coefficients constants}

\section{Systèmes linéaires à coefficients variables}

\section{Systèmes d'équations différentielles ordinaires non linéaires}

\subsection*{3.1}
Supposons sans perte de généralité que $x \leq y$ pour que $[x,y]$ ait du sens
\begin{enumerate}
  \item Lipchitzienne avec $L = 1$ car, comme
    $|\sin t| \leq 1$ $\forall t \in \mathbb{R}$,
    \[ |f(t,y) - f(t,x)| \leq |\cos y - \cos x| \]
    et par Taylor, $\exists c \in [x,y]$ tel que
    \[ \cos y = \cos x - \sin c (y - x) \]
    donc
    \[ |f(t,y) - f(t,x)| \leq |\sin c| \cdot |y - x| \leq |y - x|. \]
  \item Pas lipschitzienne car
    \[ |f(t,y) - f(t,x)| = |t|\cdot||y| - |x|| \]
    et comme $L$ ne peut pas dépendre de $t$ et que $|t|$ peut être aussi
    grand que l'on veut, il n'existe pas de constante $L$.
  \item Lipschitzienne avec $L = 1$ car
    \[ |f(t,y) - f(t,x)| = ||y| - |x|| \leq |y - x|. \]
  \item Lipschitzienne avec $L = 1$ car
    \[ |f(t,y) - f(t,x)| = |t|\cdot||y| - |x|| \leq \cdot |y - x|. \]
  \item $\sqrt{t}$ défini pour tout $t \in [-1,1]$ ?
  \item Pas lipschitzienne avec car
    \[ |f(t,y) - f(t,x)| = |t|\cdot|\sqrt{y} - \sqrt{x}| \]
    et par Taylor, $\exists c \in [x,y]$ tel que
    \[ \sqrt{y} = \sqrt{x} + \frac{1}{2\sqrt{c}}(y-x) \]
    d'où
    \[ |f(t,y) - f(t,x)| = |t|\frac{1}{2\sqrt{c}}|y-x|. \]
    En prenant $t \neq 0$ et en faisant tendre $x$ et $y$ vers 0,
    $\frac{1}{c}$ devient aussi grand que l'on veut donc $L$ n'existe pas.
  \item Idem, ne serait-ce pas $f:[-1,1]\times[1,\infty[$ ?
  \item Lipschitzienne avec $L = 1$ car $\forall t \in [0,1]$,
    \[ |f(t,y) - f(t,x)| = ||y| - |x|| \leq |y - x| \]
    et $\forall t \in [-1,0[$,
    \[ |f(t,y) - f(t,x)| = |-|y| + |x|| \leq |y - x|. \]
  \item Lipchitzienne avec $L = 1$ car, par Taylor,
    $\exists c \in [x,y]$ tel que
    \[ \arctan y = \arctan x + \frac{1}{1+c^2} (y - x) \]
    donc
    \[ |f(t,y) - f(t,x)| \leq \frac{1}{1+c^2} \cdot |y - x| \leq |y - x|. \]
\end{enumerate}

\subsection*{3.6}
\begin{enumerate}
  \item Oui, par l'unicité globale car comme
    $f(t,x) = 0$ est de classe $C^1$ en espace,
    $f$ et localement lipschitzien.
  \item Oui, par l'unicité globale car comme
    $f(t,x) = x^2$ est de classe $C^1$ en espace,
    $f$ et localement lipschitzien.
  \item Non, en intégrant on obtient $u(t)\ln u(t) + u(t) = t$.
    On trouve deux solutions à ça qui sont ?
  \item Oui, par l'unicité globale car comme
    $f(t,x) = \sin(x)$ est de classe $C^1$ en espace,
    $f$ est localement lipschitzien.
  \item Oui, la dérivée en espace de $f(t,x) = \sqrt[3]{x^2}$ est
    $D_x f(t,x) = \frac{1}{3\sqrt[3]{x}}$ qui est continue sauf en $x = 0$.
    Seulement, tout n'est pas perdu car on intégrant, on trouve
    $3\sqrt[3]{u(t)} = t$ qui ne comporte comme unique solution
    $u(t) = \frac{t^3}{27}$.
\end{enumerate}

\subsection*{3.16}
\begin{enumerate}
  \item
  \item
  \item
  \item
  \item
  \item
  \item
  \item
  \item
\end{enumerate}

\subsection*{3.18}
\begin{enumerate}
  \item
  \item
  \item
  \item
  \item
\end{enumerate}

\section{Stabilité de systèmes d'équations différentielles autonomes}

\section{Problèmes aux limites pour les
systèmes d'équations différentielles linéaires}

\subsection*{5.3}
\begin{enumerate}
  \item
    Reformulons le problème comme suit
    \begin{align*}
    \begin{pmatrix}
      u_1\\u_2
    \end{pmatrix}' & =
    \begin{pmatrix}
      0 & 1\\-\lambda^2 & 0
    \end{pmatrix}
    \begin{pmatrix}
      u_1\\u_2
    \end{pmatrix}
    +
    \begin{pmatrix}
      0\\f
    \end{pmatrix}\\
    0 & =  u_2(1) - u_1(1)\\
    u_1(-1) - u_2(-1) & = 0.
    \end{align*}

    Ce système possède une solution
    ssi $\forall v: [-1;1] \to \mathbb{R}^2$ tel que
    \[\begin{pmatrix}
      v_1\\v_2
    \end{pmatrix}' =
    \begin{pmatrix}
      0 & \lambda^2\\-1 & 0
    \end{pmatrix} 
    \begin{pmatrix}v_1\\v_2
    \end{pmatrix}\]

    et $(\alpha|v(-1)) = (\beta|v(1))$ $\forall \alpha,\beta \in\mathbb{R}^2$
    tels que $B_{-1}(\alpha) = B_1(\beta)$
    avec $B_{-1}[\alpha] =
    \begin{pmatrix}
      0\\\alpha_1-\alpha_2
    \end{pmatrix}$ et
    $B_1 =
    \begin{pmatrix}
      \beta_2-\beta_1\\0
    \end{pmatrix}$
    on a
    \[ \int_{-1}^1 f(s)v_2(s) \dif s = 0 \].


    Donc $\alpha_1 = \alpha_2$ et $\beta_1 = \beta_2$.
    Les conditions aux limites pour $v$ deviennent
    \[\alpha_1(v_1(-1) + v_2(-1)) = \beta_1(v_1(1) + v_2(1))\]
    $\forall \alpha_1,\beta_1 \in\mathbb{R}$.

    On a 2 conditions aux limites linéairement indépendantes
    \begin{align*}
      v_1(-1) + v_2(-1) & = 0\\
      v_1(1) + v_2(1) & = 0.
    \end{align*}

    \paragraph{Résolution}
    $v_2'' = -v_1' = -\lambda^2 v_2$
    Si $\lambda \neq 0$,
    $v_2(t) = A \cos(\lambda t) + B\sin(\lambda t)$ où $A, B \in \mathbb{R}$
    \begin{align*}
      v_1(t) & = -v_2'(t)\\
      & = A\lambda\sin(\lambda t) - B\lambda\cos(\lambda t)
    \end{align*}
    Si on explicite les conditions aux limites, on a
    \begin{align}
      \label{eq:cl1}
      A \cos\lambda - B \sin\lambda - A\lambda \sin\lambda - B\lambda\cos\lambda & = 0\\
      \label{eq:cl2}
      A \cos\lambda + B \sin\lambda + A\lambda \sin\lambda - B\lambda\cos\lambda & = 0.
    \end{align}
    Par $\eqref{eq:cl1} + \eqref{eq:cl2}$,
    \[ 2A\cos\lambda - 2B\lambda\cos\lambda = 0 \]
    et par $\eqref{eq:cl1} - \eqref{eq:cl2}$,
    \[ 2B\sin\lambda + 2A\lambda\sin\lambda = 0 \]
    d'où
    \begin{align*}
      \cos\lambda (A - B\lambda) & = 0\\
      \sin\lambda (B + A\lambda) & = 0.
    \end{align*}

    On doit considérer ici plusieurs cas
    \begin{itemize}
      \item Si $\lambda \neq k\frac{\pi}{2}$, $k \in \mathbb{Z}$,
        $A - B\lambda = 0$ et $B + A\lambda = 0$ donc $A = B = 0$.

      \item Si $\lambda = k\pi$, $k \in \mathbb{Z}_0$, $A = B\lambda$,
        donc
        \[ 0 = \int_{-1}^1 f(t) (\lambda \cos(\lambda t) + \sin(\lambda t)) \dif t. \]

      \item Si $\lambda = \frac{\pi}{2} + k\pi$, $k \in \mathbb{Z}$, $B = -A\lambda$,
        donc
        \[ 0 = \int_{-1}^1 f(t) (\cos(\lambda t) - \lambda\sin(\lambda t)) \dif t. \]
    \end{itemize}

    Si $\lambda = 0$,
    \begin{align*}
      v_2(t) & = A + Bt\\
      v_1(t) & = -B
    \end{align*}
    et par les conditions aux limites
    \begin{align*}
      A + B - B & = 0\\
      A - B - B & = 0.
    \end{align*}
    Donc $A = B = 0$, du coup, pas de contrainte sur $f$.
\end{enumerate}

\end{document}
