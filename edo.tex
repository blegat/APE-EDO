\documentclass[fontsize=12pt]{article}

\usepackage[utf8x]{inputenc}
\usepackage[frenchb]{babel}
\usepackage[T1]{fontenc}
\usepackage{lmodern}
\usepackage[top=2cm, bottom=2cm, left=2.5cm, right=2.5cm]{geometry}
\usepackage{graphicx}
\usepackage{epstopdf}
\usepackage{caption}
\usepackage{subcaption}
\usepackage{multirow}

% Math symbols
\usepackage{amsmath}
\usepackage{amssymb}
\usepackage{amsthm}

% Numbers and units
\usepackage[squaren, Gray]{SIunits}
\usepackage{sistyle}
\usepackage[autolanguage]{numprint}
%\usepackage{numprint}
\newcommand\si[2]{\numprint[#2]{#1}}
\newcommand\np[1]{\numprint{#1}}

\DeclareMathOperator{\newdiff}{d} % use \dif instead
\newcommand{\dif}{\newdiff\!}
\newcommand{\fpart}[2]{\frac{\partial #1}{\partial #2}}
\newcommand{\ffpart}[2]{\frac{\partial^2 #1}{\partial #2^2}}
\newcommand{\fdpart}[3]{\frac{\partial^2 #1}{\partial #2\partial #3}}
\newcommand{\fdif}[2]{\frac{\dif #1}{\dif #2}}
\newcommand{\ffdif}[2]{\frac{\dif^2 #1}{\dif #2^2}}
\newcommand{\constant}{\ensuremath{\mathrm{cst}}}

\newcommand{\vmin}{V_\mathrm{min}}
\newcommand{\vmax}{V_\mathrm{max}}
\newcommand{\tmax}{T_\mathrm{max}}
\newcommand{\dmax}{D_\mathrm{max}}
\newcommand{\tdmin}{TD_\mathrm{min}}
\newcommand{\vtmax}{VT_\mathrm{max}}
\newcommand{\vdmax}{VD_\mathrm{max}}
\newcommand{\mmax}{M_\mathrm{max}}
\newcommand{\cjour}{C_\mathrm{jour}}

\usepackage{xparse}% for using parameters at the end block

% Inclure une image
\NewDocumentEnvironment{image}{mm}
{\begin{figure}[!ht]\begin{center}}
{\caption{#2}\label{#1}\end{center}\end{figure}}

\newcommand{\showimage}[3]
{\begin{image}{fig:#2}{#3}\centering
\includegraphics[width=\textwidth]{#1}\end{image}}

% Inclure in plot matlab
\usepackage{xargs}
\newcommand{\matlabplot}[2]
{\showimage{Images/#1.eps}{#1}{#2 Généré par le listing~\ref{lst:#1}}.}
\newcommand{\matlabcode}[2]
{\lstinputlisting[caption={Contenu du fichier \lstinline{#1.m}
  contenant l'implementation de la fonction \lstinline{#1}.
#2},label={lst:#1}]
{MATLAB/#1.m}}

\usepackage[usenames,dvipsnames,svgnames,table]{xcolor}
\usepackage{listings}
\definecolor{dkgreen}{rgb}{0.25,0.7,0.35}
\definecolor{dkred}{rgb}{0.7,0,0}
\lstset{language={Matlab},numbers=left,numberstyle=\tiny\color{gray},
keywordstyle=\bfseries\color{dkred},frame=single,
commentstyle=\color{gray}=small,stringstyle=\color{dkgreen},
basicstyle=\rm\footnotesize\ttfamily} % basicstyle=\rm\footnotesize

\newcommand{\matlab}{\textsc{Matlab}}

\usepackage{wasysym}

\usepackage{bm}

\title{Correctif EDO}
\author{Benoît Legat \and Julien Vaes}

\begin{document}

\maketitle

Les corrections de ce correctif seront soumises au prof d'EDO pour
être rajoutée dans le syllabus de l'an prochain.
En attendant, elles sont publiées par ce biais ci.
Vous êtes bien entendu invités à en ajouter.

\section{Systèmes linéaires à coefficients constants}

\subsection*{1.18}
\begin{enumerate}
  \item[a)]
    \begin{equation}
      w''(t)-4w'(t)-5w(t)=f(t)
    \end{equation}

    Posons
    \begin{equation}
      \bm{u}(t) = \left(
        \begin{matrix}
          w(t)\\w'(t)
        \end{matrix}
      \right)
    \end{equation}
    Nous avons le système suivant
    \begin{equation}
      \bm{u}'(t) = \left(
        \begin{matrix}
          0&1 \\
          4&5 \\
        \end{matrix}
      \right) \cdot \bm{u}+\left(
        \begin{matrix}
          0 \\
          f(t) \\
        \end{matrix}
      \right)
    \end{equation}

    Trouvons donc l'expression de $e^{tA}$. Nous savons que $\frac{de^{tA}}{dt} = Ae^{tA}$. Nous savons aussi que
    \begin{equation}
      e^{0A} = I
    \end{equation}

    Posons
    \begin{equation}
      e^{tA}= \left(
        \begin{matrix}
          a&b \\
          c&d \\
        \end{matrix}
      \right)
    \end{equation}
    Pour trouver $e^{tA}$ nous devons donc résoudre
    \begin{eqnarray*}
      \left(
        \begin{matrix}
          a'&b' \\
          c'&d' \\
        \end{matrix}
      \right) &=&\left(
        \begin{matrix}
          0&1 \\
          4&5 \\
        \end{matrix}
      \right) \cdot\left( \begin{matrix}
          a&b \\
          c&d \\
        \end{matrix}
      \right)\\
      &=& \left(
      \begin{matrix}
        c&d \\
        4a+5c&4b+5d \\
      \end{matrix}
    \right)
  \end{eqnarray*}

  Nous avons donc que
  \begin{equation}
    c''(t) -5c'(t) -4c(t)=0
  \end{equation}

  Donc
  \begin{equation}
    c(t) = C_1 e^{\frac{5+\sqrt{41}}{2}} + C_2 e^{\frac{5-\sqrt{41}}{2}}
  \end{equation}



\item[b)]
\end{enumerate}


\section{Systèmes linéaires à coefficients variables}

\subsection*{2.15}
Nous avons l'équation suivante:
\begin{equation} \label{eq_2.15_1}
u''(t) = \frac{u'(t)}{t} + \left( 1 - \frac{3}{t} \right) \cdot u(t)  \hspace{1 cm } t>0
\end{equation}

ainsi qu'une solution
\begin{equation}
u_1(t) = t^2 e^{-t}
\end{equation}

Posons
\begin{eqnarray}
\bm{u_1}(t) &=& (u_1(t)  \hspace{0.3cm}u_1'(t))^T\\
&=&(t^2 e^{-t} \hspace{0.5cm} 2t e^{-t} -t^2 e^{-t} )^T
\end{eqnarray}



Nous pouvons mettre ce problème sous le forme matricielle:
\begin{eqnarray*}
\bm{v}' &=& \bm{A}[\bm{v}]\\
\bm{v}'&=&\left(
\begin {matrix}
 0&1 \\
 1-\frac{3}{t}&\frac{1}{t} \\
\end{matrix}
\right) \cdot \bm{v}
\end{eqnarray*}

Nous savons que la seconde solution de l'équation \ref{eq_2.15_1} est de la forme
\begin{equation}
\bm{u_2}(t) = \alpha(t) \bm{u_1}(t) + \bm{w}(t)
\end{equation}

où nous avons un vecteur $\bm{e}$ tel que
\begin{eqnarray*}
(\bm{e}|\bm{u_1}(t)) &\neq& 0\\
(\bm{e}|\bm{w}(t))&=&0
\end{eqnarray*}

Prenons $\bm{e}=(1\hspace{0.3cm}0)^T$ nous avons donc que $\bm{w}(t) = (0\hspace{0.3cm}\beta(t))^T$.

Nous savons que la dérivée de $\bm{u_2}$ doit être égale à sa dérivée via le système matriciel. Nous obtenons donc le système suivant:
\begin{eqnarray}
\bm{A}[\alpha(t) \bm{u_1}(t)]+\bm{A}[(0 \hspace{0.3 cm} \beta(t))^T] &=& \bm{A}[\alpha(t)\bm{u_1}(t)] + \alpha'(t)\bm{u_1}(t)+\beta'(t)\cdot (0 \hspace{0.3 cm} 1)^T\\
\bm{A}[(0 \hspace{0.3 cm} \beta(t))^T] &=&  \alpha'(t)\bm{u_1}(t)+\beta'(t)\cdot (0 \hspace{0.3 cm} 1)^T
\end{eqnarray}

 Ce qui donne les deux équations suivantes
 \begin{eqnarray}
\beta(t)&=&\alpha'(t)t^2e^{-t}\\
\frac{\beta(t)}{t}&=&\alpha'(t)\cdot \left( 2t e^{-t} -t^2 e^{-t}  \right) +\beta'(t)
\end{eqnarray}

On obtient donc que
\begin{equation}
\alpha'(t) = \frac{\beta(t)}{t^2 e^{-t}} \label{eq_2.15_ab}
\end{equation}

En remplaçant cela dans la seconde équation du système, nous obtenons:
\begin{eqnarray*}
\frac{\beta(t)}{t}&=&\frac{\beta(t)}{t^2 e^{-t}}\cdot \left( 2t e^{-t} -t^2 e^{-t}  \right) +\beta'(t)\\
\frac{\beta(t)}{t}&=&\frac{\beta(t)}{t}\cdot \left( 2-t  \right) +\beta'(t)\\
0&=& \frac{\beta(t)}{t}\cdot \left( 1-t  \right)+\beta'(t)\\
\frac{\beta'(t)}{\beta(t)}&=&\frac{t-1}{t} = 1-\frac{1}{t}\\
\int_1^t \frac{\beta'(t)}{\beta(t)} &=& \int_1^t1-\frac{1}{t}\\
\ln{\beta(t)} &=& t-\ln t + 1
\end{eqnarray*}
On peut supprimer les constant car après dérivation elles disparaissent. On obtient donc que
\begin{equation}
\beta(t) = \frac{e^{t}}{t}
\end{equation}

De part l'équation \ref{eq_2.15_ab}, on trouve
\begin{equation}
\alpha(t)= \left( \int_{t_0}^t \frac{e^{2t}}{t^3} \right)
\end{equation}

On trouve donc comme seconde solution:
\begin{equation}
v_2(t) = \left( \int_{t_0}^t \frac{e^{2t}}{t^3} \right) \cdot t^2 e^{-t}
\end{equation}


\section{Systèmes d'équations différentielles ordinaires non linéaires}

\subsection*{3.1}
Supposons sans perte de généralité que $x \leq y$ pour que $[x,y]$ ait du sens
\begin{enumerate}
  \item Lipchitzienne avec $L = 1$ car, comme
    $|\sin t| \leq 1$ $\forall t \in \mathbb{R}$,
    \[ |f(t,y) - f(t,x)| \leq |\cos y - \cos x| \]
    et par Taylor, $\exists c \in [x,y]$ tel que
    \[ \cos y = \cos x - \sin c (y - x) \]
    donc
    \[ |f(t,y) - f(t,x)| \leq |\sin c| \cdot |y - x| \leq |y - x|. \]
  \item Pas lipschitzienne car
    \[ |f(t,y) - f(t,x)| = |t|\cdot||y| - |x|| \]
    et comme $L$ ne peut pas dépendre de $t$ et que $|t|$ peut être aussi
    grand que l'on veut, il n'existe pas de constante $L$.
  \item Lipschitzienne avec $L = 1$ car
    \[ |f(t,y) - f(t,x)| = ||y| - |x|| \leq |y - x|. \]
  \item Lipschitzienne avec $L = 1$ car
    \[ |f(t,y) - f(t,x)| = |t|\cdot||y| - |x|| \leq \cdot |y - x|. \]
  \item $\sqrt{t}$ défini pour tout $t \in [-1,1]$ ?
  \item Pas lipschitzienne avec car
    \[ |f(t,y) - f(t,x)| = |t|\cdot|\sqrt{y} - \sqrt{x}| \]
    et par Taylor, $\exists c \in [x,y]$ tel que
    \[ \sqrt{y} = \sqrt{x} + \frac{1}{2\sqrt{c}}(y-x) \]
    d'où
    \[ |f(t,y) - f(t,x)| = |t|\frac{1}{2\sqrt{c}}|y-x|. \]
    En prenant $t \neq 0$ et en faisant tendre $x$ et $y$ vers 0,
    $\frac{1}{c}$ devient aussi grand que l'on veut donc $L$ n'existe pas.
  \item Idem, ne serait-ce pas $f:[-1,1]\times[1,\infty[$ ?
  \item Lipschitzienne avec $L = 1$ car $\forall t \in [0,1]$,
    \[ |f(t,y) - f(t,x)| = ||y| - |x|| \leq |y - x| \]
    et $\forall t \in [-1,0[$,
    \[ |f(t,y) - f(t,x)| = |-|y| + |x|| \leq |y - x|. \]
  \item Lipchitzienne avec $L = 1$ car, par Taylor,
    $\exists c \in [x,y]$ tel que
    \[ \arctan y = \arctan x + \frac{1}{1+c^2} (y - x) \]
    donc
    \[ |f(t,y) - f(t,x)| \leq \frac{1}{1+c^2} \cdot |y - x| \leq |y - x|. \]
\end{enumerate}

\subsection*{3.6}
\begin{enumerate}
  \item Oui, par l'unicité globale car comme
    $f(t,x) = 0$ est de classe $C^1$ en espace,
    $f$ et localement lipschitzien.
  \item Oui, par l'unicité globale car comme
    $f(t,x) = x^2$ est de classe $C^1$ en espace,
    $f$ et localement lipschitzien.
  \item Non, en intégrant on obtient $u(t)\ln u(t) + u(t) = t$.
    On trouve deux solutions à ça qui sont ?
  \item Oui, par l'unicité globale car comme
    $f(t,x) = \sin(x)$ est de classe $C^1$ en espace,
    $f$ est localement lipschitzien.
  \item Non, la dérivée en espace de $f(t,x) = \sqrt[3]{x^2}$ est
    $D_x f(t,x) = \frac{1}{3\sqrt[3]{x}}$ qui est continue sauf en $x = 0$.
    Ça ne nous permet tout de même pas de conclure car c'est suffisant à
    l'unicité mais pas nécessaire.
    On pourrait croire qu'elle est unique car en intégrant, on trouve
    $3\sqrt[3]{u(t)} = t$ qui ne comporte comme unique solution
    $u(t) = \frac{t^3}{27}$ mais ce faisant, on divise par $u(t)$ et
    $u(t) = 0$ pour tout $t$ est aussi une solution.
\end{enumerate}

\subsection*{3.16}
\begin{enumerate}
  \item
  \item
  \item
  \item
  \item
  \item
  \item
  \item
  \item
\end{enumerate}

\subsection*{3.18}
\begin{enumerate}
  \item Oui. Commençons par prouver qu'il existe une courbe intégrale maximale.
    On peut réécrir le problème comme suit
    \begin{align*}
      \begin{pmatrix}
        u'(t)\\
        v'(t)
      \end{pmatrix} & =
      \begin{pmatrix}
        v(t)\\
        \sin(u(t))
      \end{pmatrix}.
    \end{align*}

    Soit $f:\mathbb{R}\times\mathbb{R}^2\to\mathbb{R}^2:
    (t,u,v)\mapsto(v,\sin u)$, on voit que $f$ est continue et que
    $f$ est localement lipschitzienne car
    \[ D_uf(t,u,v) =
    \begin{pmatrix}
      0 & 1\\
      \cos u & 0
    \end{pmatrix} \]
    est continue.

    Par le théorème d'existence maximale, il existe donc une courbe intégrale
    maximale et elle est unique.

    Soit $u:I\to\mathbb{R}$ cette courbe intégrale maximale telle que
    $u(0) = u_0$ et $v(0) = v_0$, montrons que $I = \mathbb{R}$.
    Soit $u$ une courbe intégrale maximale.
    On remarque que $|u''(t)| \leq 1$.
    Du coup,
    \begin{align*}
      |u'(t) - v_0| & \leq |t - t_0|\\
      |u'(t)| & \leq |v_0| + |t - t_0|\\
      |u(t) - u_0| & \leq |v_0|\cdot|t - t_0| + \frac{|t - t_0|^2}{2}\\
      |u(t)| & \leq |u_0| + |v_0|\cdot|t - t_0| + \frac{|t - t_0|^2}{2}.
    \end{align*}
    Soit $T > 0$. Considérons
    \[ K_T = \{(t,u) \in \mathbb{R}^2|
    t \in [t_0-T,t_0+T] \land |u| \leq
    |u_0| + |v_0|\cdot|t - t_0| + \frac{|t - t_0|^2}{2}\}. \]
    On remarque que $K_T$ est compact et comme $u$ est une courbe intégrale
    maximale,
    \begin{align*}
      \sup\{t \in I : (t,u(t)) \in K_T\} & \in I\\
      \inf\{t \in I : (t,u(t)) \in K_T\} & \in I
    \end{align*}
    or, par construction de $K_T$,
    \begin{align*}
      \sup\{t \in I : (t,u(t)) \in K_T\} & = t_0 + T\\
      \inf\{t \in I : (t,u(t)) \in K_T\} & = t_0 - T.
    \end{align*}
    Donc $[t_0-T;t_0+T] \subseteq I$.
    Puisque ceci est vrai $\forall T > 0$, $I = \mathbb{R}$.
  \item
    Idem \smiley
  \item
    Commençons par prouver qu'il existe une courbe intégrale maximale
    et qu'elle est unique.
    En effet, si on montre que pour certains $u_0,v_0$, une courbe intégrale
    n'est pas maximale, si on ne sait pas qu'elle est unique,
    on ne prouve pas qu'il y en a pas pour ces $u_0,v_0$
    définis sur $\mathbb{R}$.
    On peut réécrir le problème comme suit
    \begin{align*}
      \begin{pmatrix}
        u'(t)\\
        v'(t)
      \end{pmatrix} & =
      \begin{pmatrix}
        v(t)\\
        u(t)^3
      \end{pmatrix}.
    \end{align*}

    Soit $f:\mathbb{R}\times\mathbb{R}^2\to\mathbb{R}^2:
    (t,u,v)\mapsto(v,\sin u)$, on voit que $f$ est continue et que
    $f$ est localement lipschitzienne car
    \[ D_uf(t,u,v) =
      \begin{pmatrix}
        0 & 1\\
        3u^2 & 0
    \end{pmatrix} \]
    est continue.

    Par le théorème d'existence maximale, il existe donc une courbe intégrale
    maximale et elle est unique.

    Soit $u:I\to\mathbb{R}$ cette courbe intégrale maximale telle que

    On remarque que $\frac{\sqrt{2}}{t-1}$ est solution et que cette solution
    est maximale sur $I = ]-\infty,1[$.
  \item
  \item
\end{enumerate}

\section{Stabilité de systèmes d'équations différentielles autonomes}

\section{Problèmes aux limites pour les
systèmes d'équations différentielles linéaires}

\subsection*{5.3}
\begin{enumerate}
  \item
    Reformulons le problème comme suit
    \begin{align*}
    \begin{pmatrix}
      u_1\\u_2
    \end{pmatrix}' & =
    \begin{pmatrix}
      0 & 1\\-\lambda^2 & 0
    \end{pmatrix}
    \begin{pmatrix}
      u_1\\u_2
    \end{pmatrix}
    +
    \begin{pmatrix}
      0\\f
    \end{pmatrix}\\
    0 & =  u_2(1) - u_1(1)\\
    u_1(-1) - u_2(-1) & = 0.
    \end{align*}

    Ce système possède une solution
    ssi $\forall v: [-1;1] \to \mathbb{R}^2$ tel que
    \[\begin{pmatrix}
      v_1\\v_2
    \end{pmatrix}' =
    \begin{pmatrix}
      0 & \lambda^2\\-1 & 0
    \end{pmatrix}
    \begin{pmatrix}v_1\\v_2
    \end{pmatrix}\]

    et $(\alpha|v(-1)) = (\beta|v(1))$ $\forall \alpha,\beta \in\mathbb{R}^2$
    tels que $B_{-1}(\alpha) = B_1(\beta)$
    avec $B_{-1}[\alpha] =
    \begin{pmatrix}
      0\\\alpha_1-\alpha_2
    \end{pmatrix}$ et
    $B_1 =
    \begin{pmatrix}
      \beta_2-\beta_1\\0
    \end{pmatrix}$
    on a
    \[ \int_{-1}^1 f(s)v_2(s) \dif s = 0 \].


    Donc $\alpha_1 = \alpha_2$ et $\beta_1 = \beta_2$.
    Les conditions aux limites pour $v$ deviennent
    \[\alpha_1(v_1(-1) + v_2(-1)) = \beta_1(v_1(1) + v_2(1))\]
    $\forall \alpha_1,\beta_1 \in\mathbb{R}$.

    On a 2 conditions aux limites linéairement indépendantes
    \begin{align*}
      v_1(-1) + v_2(-1) & = 0\\
      v_1(1) + v_2(1) & = 0.
    \end{align*}

    \paragraph{Résolution}
    $v_2'' = -v_1' = -\lambda^2 v_2$
    Si $\lambda \neq 0$,
    $v_2(t) = A \cos(\lambda t) + B\sin(\lambda t)$ où $A, B \in \mathbb{R}$
    \begin{align*}
      v_1(t) & = -v_2'(t)\\
      & = A\lambda\sin(\lambda t) - B\lambda\cos(\lambda t)
    \end{align*}
    Si on explicite les conditions aux limites, on a
    \begin{align}
      \label{eq:cl1}
      A \cos\lambda - B \sin\lambda - A\lambda \sin\lambda - B\lambda\cos\lambda & = 0\\
      \label{eq:cl2}
      A \cos\lambda + B \sin\lambda + A\lambda \sin\lambda - B\lambda\cos\lambda & = 0.
    \end{align}
    Par $\eqref{eq:cl1} + \eqref{eq:cl2}$,
    \[ 2A\cos\lambda - 2B\lambda\cos\lambda = 0 \]
    et par $\eqref{eq:cl1} - \eqref{eq:cl2}$,
    \[ 2B\sin\lambda + 2A\lambda\sin\lambda = 0 \]
    d'où
    \begin{align*}
      \cos\lambda (A - B\lambda) & = 0\\
      \sin\lambda (B + A\lambda) & = 0.
    \end{align*}

    On doit considérer ici plusieurs cas
    \begin{itemize}
      \item Si $\lambda \neq k\frac{\pi}{2}$, $k \in \mathbb{Z}$,
        $A - B\lambda = 0$ et $B + A\lambda = 0$ donc $A = B = 0$.

      \item Si $\lambda = k\pi$, $k \in \mathbb{Z}_0$, $A = B\lambda$,
        donc
        \[ 0 = \int_{-1}^1 f(t) (\lambda \cos(\lambda t) + \sin(\lambda t)) \dif t. \]

      \item Si $\lambda = \frac{\pi}{2} + k\pi$, $k \in \mathbb{Z}$, $B = -A\lambda$,
        donc
        \[ 0 = \int_{-1}^1 f(t) (\cos(\lambda t) - \lambda\sin(\lambda t)) \dif t. \]
    \end{itemize}

    Si $\lambda = 0$,
    \begin{align*}
      v_2(t) & = A + Bt\\
      v_1(t) & = -B
    \end{align*}
    et par les conditions aux limites
    \begin{align*}
      A + B - B & = 0\\
      A - B - B & = 0.
    \end{align*}
    Donc $A = B = 0$, du coup, pas de contrainte sur $f$.
\end{enumerate}

\end{document}
