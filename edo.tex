\documentclass[fontsize=12pt]{article}

\usepackage[utf8x]{inputenc}
\usepackage[frenchb]{babel}
\usepackage[T1]{fontenc}
\usepackage{lmodern}
\usepackage[top=2cm, bottom=2cm, left=2.5cm, right=2.5cm]{geometry}
\usepackage{graphicx}
\usepackage{epstopdf}
\usepackage{caption}
\usepackage{subcaption}
\usepackage{multirow}

% Math symbols
\usepackage{amsmath}
\usepackage{amssymb}
\usepackage{amsthm}

% Numbers and units
\usepackage[squaren, Gray]{SIunits}
\usepackage{sistyle}
\usepackage[autolanguage]{numprint}
%\usepackage{numprint}
\newcommand\si[2]{\numprint[#2]{#1}}
\newcommand\np[1]{\numprint{#1}}

\DeclareMathOperator{\newdiff}{d} % use \dif instead
\newcommand{\dif}{\newdiff\!}
\newcommand{\fpart}[2]{\frac{\partial #1}{\partial #2}}
\newcommand{\ffpart}[2]{\frac{\partial^2 #1}{\partial #2^2}}
\newcommand{\fdpart}[3]{\frac{\partial^2 #1}{\partial #2\partial #3}}
\newcommand{\fdif}[2]{\frac{\dif #1}{\dif #2}}
\newcommand{\ffdif}[2]{\frac{\dif^2 #1}{\dif #2^2}}
\newcommand{\constant}{\ensuremath{\mathrm{cst}}}

\newcommand{\vmin}{V_\mathrm{min}}
\newcommand{\vmax}{V_\mathrm{max}}
\newcommand{\tmax}{T_\mathrm{max}}
\newcommand{\dmax}{D_\mathrm{max}}
\newcommand{\tdmin}{TD_\mathrm{min}}
\newcommand{\vtmax}{VT_\mathrm{max}}
\newcommand{\vdmax}{VD_\mathrm{max}}
\newcommand{\mmax}{M_\mathrm{max}}
\newcommand{\cjour}{C_\mathrm{jour}}

\usepackage{xparse}% for using parameters at the end block

% Inclure une image
\NewDocumentEnvironment{image}{mm}
{\begin{figure}[!ht]\begin{center}}
{\caption{#2}\label{#1}\end{center}\end{figure}}

\newcommand{\showimage}[3]
{\begin{image}{fig:#2}{#3}\centering
\includegraphics[width=\textwidth]{#1}\end{image}}

% Inclure in plot matlab
\usepackage{xargs}
\newcommand{\matlabplot}[2]
{\showimage{Images/#1.eps}{#1}{#2 Généré par le listing~\ref{lst:#1}}.}
\newcommand{\matlabcode}[2]
{\lstinputlisting[caption={Contenu du fichier \lstinline{#1.m}
  contenant l'implementation de la fonction \lstinline{#1}.
#2},label={lst:#1}]
{MATLAB/#1.m}}

\usepackage[usenames,dvipsnames,svgnames,table]{xcolor}
\usepackage{listings}
\definecolor{dkgreen}{rgb}{0.25,0.7,0.35}
\definecolor{dkred}{rgb}{0.7,0,0}
\lstset{language={Matlab},numbers=left,numberstyle=\tiny\color{gray},
keywordstyle=\bfseries\color{dkred},frame=single,
commentstyle=\color{gray}=small,stringstyle=\color{dkgreen},
basicstyle=\rm\footnotesize\ttfamily} % basicstyle=\rm\footnotesize

\newcommand{\matlab}{\textsc{Matlab}}

\title{Correctif EDO}
\author{Benoît Legat}

\begin{document}

\maketitle

Les corrections de ce correctif seront soumises au prof d'EDO pour
être rajoutée dans le syllabus de l'an prochain.
En attendant, elles sont publiées par ce biais ci.
Vous êtes bien entendu invités à en ajouter.

\section*{5.3.1}
Reformulons le problème comme suit
\begin{align*}
\begin{pmatrix}
  u_1\\u_2
\end{pmatrix}' & =
\begin{pmatrix}
  0 & 1\\-\lambda^2 & 0
\end{pmatrix}
\begin{pmatrix}
  u_1\\u_2
\end{pmatrix}
+
\begin{pmatrix}
  0\\f
\end{pmatrix}\\
0 & =  u_2(1) - u_1(1)\\
u_1(-1) - u_2(-1) & = 0.
\end{align*}

Ce système possède une solution
ssi $\forall v: [-1;1] \to \mathbb{R}^2$ tel que
\[\begin{pmatrix}
  v_1\\v_2
\end{pmatrix}' =
\begin{pmatrix}
  0 & \lambda^2\\-1 & 0
\end{pmatrix} 
\begin{pmatrix}v_1\\v_2
\end{pmatrix}\]

et $(\alpha|v(-1)) = (\beta|v(1))$ $\forall \alpha,\beta \in\mathbb{R}^2$
tels que $B_{-1}(\alpha) = B_1(\beta)$
avec $B_{-1}[\alpha] =
\begin{pmatrix}
  0\\\alpha_1-\alpha_2
\end{pmatrix}$ et
$B_1 =
\begin{pmatrix}
  \beta_2-\beta_1\\0
\end{pmatrix}$
on a
\[ \int_{-1}^1 f(s)v_2(s) \dif s = 0 \].


Donc $\alpha_1 = \alpha_2$ et $\beta_1 = \beta_2$.
Les conditions aux limites pour $v$ deviennent
\[\alpha_1(v_1(-1) + v_2(-1)) = \beta_1(v_1(1) + v_2(1))\]
$\forall \alpha_1,\beta_1 \in\mathbb{R}$.

On a 2 conditions aux limites linéairement indépendantes
\begin{align*}
  v_1(-1) + v_2(-1) & = 0\\
  v_1(1) + v_2(1) & = 0.
\end{align*}

\paragraph{Résolution}
$v_2'' = -v_1' = -\lambda^2 v_2$
Si $\lambda \neq 0$,
$v_2(t) = A \cos(\lambda t) + B\sin(\lambda t)$ où $A, B \in \mathbb{R}$
\begin{align*}
  v_1(t) & = -v_2'(t)\\
  & = A\lambda\sin(\lambda t) - B\lambda\cos(\lambda t)
\end{align*}
Si on explicite les conditions aux limites, on a
\begin{align}
  \label{eq:cl1}
  A \cos\lambda - B \sin\lambda - A\lambda \sin\lambda - B\lambda\cos\lambda & = 0\\
  \label{eq:cl2}
  A \cos\lambda + B \sin\lambda + A\lambda \sin\lambda - B\lambda\cos\lambda & = 0.
\end{align}
Par $\eqref{eq:cl1} + \eqref{eq:cl2}$,
\[ 2A\cos\lambda - 2B\lambda\cos\lambda = 0 \]
et par $\eqref{eq:cl1} - \eqref{eq:cl2}$,
\[ 2B\sin\lambda + 2A\lambda\sin\lambda = 0 \]
d'où
\begin{align*}
  \cos\lambda (A - B\lambda) & = 0\\
  \sin\lambda (B + A\lambda) & = 0.
\end{align*}

On doit considérer ici plusieurs cas
\begin{itemize}
  \item Si $\lambda \neq k\frac{\pi}{2}$, $k \in \mathbb{Z}$,
    $A - B\lambda = 0$ et $B + A\lambda = 0$ donc $A = B = 0$.

  \item Si $\lambda = k\pi$, $k \in \mathbb{Z}_0$, $A = B\lambda$,
    donc
    \[ 0 = \int_{-1}^1 f(t) (\lambda \cos(\lambda t) + \sin(\lambda t)) \dif t. \]

  \item Si $\lambda = \frac{\pi}{2} + k\pi$, $k \in \mathbb{Z}$, $B = -A\lambda$,
    donc
    \[ 0 = \int_{-1}^1 f(t) (\cos(\lambda t) - \lambda\sin(\lambda t)) \dif t. \]
\end{itemize}

Si $\lambda = 0$,
\begin{align*}
  v_2(t) & = A + Bt\\
  v_1(t) & = -B
\end{align*}
et par les conditions aux limites
\begin{align*}
  A + B - B & = 0\\
  A - B - B & = 0.
\end{align*}
Donc $A = B = 0$, du coup, pas de contrainte sur $f$.

\end{document}
